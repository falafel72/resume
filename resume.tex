\documentclass[a4paper]{article}
    \usepackage{fullpage}
    \usepackage{amsmath}
    \usepackage{amssymb}
    \usepackage{textcomp}
    \usepackage[utf8]{inputenc}
    \usepackage[T1]{fontenc}
    \textheight=9in
    \pagestyle{empty}
    \raggedright
    \usepackage[left=0.6in,right=0.6in,bottom=0.6in,top=0.8in]{geometry}
    \usepackage{comment}
    \usepackage{hyperref}
    \usepackage{fontawesome}


   \hypersetup{
      colorlinks=true,
      linkcolor=blue,
      filecolor=magenta,
      urlcolor=black,
    }
    %\renewcommand{\encodingdefault}{cg}
%\renewcommand{\rmdefault}{lgrcmr}

\def\bull{\vrule height 0.8ex width .7ex depth -.1ex }

% DEFINITIONS FOR RESUME %%%%%%%%%%%%%%%%%%%%%%%

\newcommand{\area} [2] {
    \vspace*{-9pt}
    \begin{verse}
        \textbf{#1}   #2
    \end{verse}
}

\newcommand{\lineunder} {
    \vspace*{-8pt} \\
    \hspace*{-18pt} \hrulefill \\
}

\newcommand{\header} [1] {
    {\hspace*{-18pt}\vspace*{6pt} \textsc{#1}}
    \vspace*{-6pt} \lineunder
}

\newcommand{\employer} [3] {
    { \textbf{#1} (#2)\\ \underline{\textbf{\emph{#3}}}\\  }
}

\newcommand{\contact} [3] {
    \vspace*{-10pt}
    \begin{center}
        {\Huge \scshape {#1}}\\
        #2 \\ #3
    \end{center}
    \vspace*{-8pt}
}

\newenvironment{achievements}{
    \begin{list}
        {$\bullet$}{\topsep 0pt \itemsep -2pt}}{\vspace*{4pt}
    \end{list}
}

\newcommand{\schoolwithcourses} [4] {
    \textbf{#1} #2 $\bullet$ #3\\
    #4 \\
    \vspace*{5pt}
}

\newcommand{\school} [4] {
    \textbf{#1} #2 $\bullet$ #3\\
    #4 \\
}

\newenvironment{entry}[4][]{
  \textbf{#2} \hfill #1 \\
  \textit{#3} \hfill #4 \\
  \vspace{-2mm}
  \begin{itemize} \itemsep 0em
  }
  {
  \end{itemize}
}
% END RESUME DEFINITIONS %%%%%%%%%%%%%%%%%%%%%%%

    \begin{document}
\vspace*{-40pt}

    

%==== Profile ====%
\vspace*{-10pt}
\begin{center}
	{\Huge \scshape {Kyle Hu}}\\
  \vspace{2mm}
	\faGlobe\:San Diego, CA
  $\cdot$ \faPhone\:(925)\,303-7083
  $\cdot$ \faEnvelope\:kylehu2000@gmail.com
  $\cdot$ \faLinkedinSquare\:\href{https://www.linkedin.com/in/kshu082/}{kshu082}
  $\cdot$ \faGithubSquare\:\href{https://www.github.com/falafel72}{falafel72}
  $\cdot$ \faChain\:\href{https://falafel72.github.io}{falafel72.github.io}\\
\end{center}

\begin{comment}
\header{Objective}
Looking for an entry-level full time job where I can fully utilize my
robotics education and systems experience.
\vspace{1mm}
\end{comment}
% ==== Education ====%
\header{Education}
\vspace{1mm}
\textbf{University of California, San Diego}\hfill San Diego, CA\\
MS Intelligent Systems, Robotics and Control \textit{GPA: 3.58} \hfill Sep 2022 - Jun 2024\\
\vspace{2mm}
\textbf{University of California, San Diego}\hfill San Diego, CA\\
BS Computer Engineering \textit{GPA: 3.77} \hfill Sep 2018 - Jun 2022\\
\vspace{2mm}

% ==== Skills ====%
\header{Skills}
\vspace{2mm}
\begin{tabular}{ l l }
	Programming Languages: & C/C++, Python, Java, MATLAB, JavaScript, SystemVerilog, Haskell, Rust  \\
%	Programming Languages: & C/C++, Python, Java, JavaScript, MATLAB, Haskell, Rust  \\
	Software Tools:       & Git, Bash, Linux, STM32CubeIDE, Numpy, OpenCV, AWS, ROS, Docker, Github Actions \\
	%Libraries/Tools:       & Git, Bash, Node.js, React, Express, Flask, Spring, MongoDB, AWS, Docker, Github Actions \\
  Hardware Tools: & Multimeter, Oscilloscope, 3D printing \\
	Spoken Languages:      & English, Cantonese, Mandarin                     \\
\end{tabular}
\vspace{2mm}

%==== Experience ====%
\header{Engineering Experience}
\vspace{1mm}
\begin{entry}[San Diego, CA]{Engineers for Exploration}{Research
    Assistant, Project Co-lead}{February 2022 - Present}
	\item Designing the first citizen science platform for gathering fish population data,
    using off-the-shelf components
  \item Derived mathematical calibration procedure using Gauss-Newton optimization
    for multiple sensor alignment
  \item Planned and conducted experiments to iterate on in-field calibration procedures and verify system accuracy
	\item Architected and developed software pipeline in \textbf{Python} to automate image
    processing and sensor calibration
\end{entry}
\begin{entry}[San Diego, CA]{MaXentric Techologies}{Engineering Intern}{June
    2022 - September 2022}
	\item Implemented high-level digital signal processing algorithms in \textbf{C} for a radar-based non-contact vitals monitor
  \item Increased speed of convergence for estimate of heart and respiration
    rate to ground truth by 50\%
	\item Heavily refactored existing radar codebase to be modular and
    extensible, while adding unit tests
\end{entry}

  \begin{entry}[San Diego, CA]{Amazon}{Software Development Engineering
      Intern}{June 2021 - September 2021}
  \item Designed and implemented a favorites list widget on the grocery cart page using \textbf{AWS} and \textbf{Spring}
	\item Performed cost/benefit analysis to identify appropriate \textbf{AWS} services (Lambda w/ Dynamo DB Table)
	\item Implemented new endpoints on micro-services used in production code to host front-end
\end{entry}

\begin{comment}
\begin{entry}[San Diego, CA]{HP}{Software Engineering Intern}{June 2020 -
    September 2020}
\item Built an internal 3D measurement tool for biometrics using CGAL, VTK and Qt (\textbf{C++}).
\item Implemented multi-threaded geodesic path searching, and employed
  Levenberg-Marquardt for model alignment
\item Digitized the measurement pipeline for custom 3D insoles, reducing time from several hours to several minutes
\end{entry}
\end{comment}

% ==== Projects ====%
\header{Projects}
\vspace{1mm}

\begin{entry}{Visual SLAM}{Software}{March 2022}
\item Implemented simultaneous localization and mapping using stereo video and IMU data
\item Implemented an extended Kalman filter to more accurately predict the car position based on sensor readings
\item Written in \textbf{Python} and \textbf{Jupyter Notebook} using pure \textbf{numpy}
\end{entry}

\begin{entry}{Serial Driver}{Firmware}{February 2022 - June 2022}
\item Part of sleep timer firmware to drive a low power radio telemetry system
  for tracking collared lizards
\item Implemented a serial driver for an \textbf{STM32} chip to facilitate
  communication over \textbf{UART}
\item Implemented ring buffer abstraction for robust processing of data from multiple serial ports concurrently
\item Ring buffer memory updated using \textbf{DMA} triggered by \textbf{UART}
  interrupt to store data as quickly as possible
\end{entry}

\begin{comment}
\begin{entry}{Carbon Price Tag}{Full Stack Web Development}{February 2020}
\item Chrome extension that displays extra carbon cost of items from Amazon
  Whole Foods
\item Implemented shipping route calculation with Google Maps API and used
  \textbf{Flask} backend
\item Winner of the Sustainability vertical at HackSC 2020
\end{entry}
\end{comment}

\begin{comment}
\begin{entry}{ACM Meme Gen}{Full Stack Web Development}{June 2019 - December 2019}
\item Created a meme generator as a supplement for students to learn MongoDB, Express, React, and Node.js
\item Implemented \textbf{React} components and state management, called the
  Imgflip API for templates
\item \textbf{MongoDB} back-end used to store user-created memes for a central gallery
\end{entry}
\end{comment}


\begin{comment}
\begin{entry}{Infantry Firmware - Robomaster}{Firmware, Circuit Design}{January 2019 - August 2019}
\item Implemented control firmware on \textbf{STM32} chip for a 2-DOF turret mounted on a 4-wheel robot
\item Angle control of motors achieved using sensor data from encoders and IMU
  over \textbf{UART} and \textbf{I2C} with \textbf{PID}
\item Designed and soldered power circuit with supercaps in parallel with
  battery to circumvent power restrictions
\end{entry}
\end{comment}
\begin{comment}
\begin{entry}{Pinball Machine}{Rapid Prototyping, Computer Aided Design, Circuit
    Design}{April 2022 - June 2022}
\item Designed and built pinball machine, including subsystems for flipper and bumper mechanism
\item Modeled all components in \textbf{Solidworks}, manufactured primarily with \textbf{3D printing} and \textbf{laser cutting}
\item Designed circuitry to control flippers on button press, and bumpers that
  trigger (via piezoelectric) when hit
\item Wrote software to run on an Arduino to control all actuators and keep
  track of game state
\end{entry}


\end{entry}
\end{comment}



%==== Publications ====%

%==== Communication ====%
\header{Communication}
\vspace{1mm}

\begin{entry}{ECE 15 (Engineering Computation)}{Teaching Assistant}{September 2022 - Present}
  \item Guided students through fundamentals of \textbf{C},
    including variables, control flow, pointers, and process memory 
  \item Held office hours, provided assistance with assignments, and graded exams for up to 250 students per class
\end{entry}

\begin{entry}{ACM UCSD}{Cofounder \& Technical Director}{May 2019 - June 2021}
  \item Co-founded the largest engineering organization at UCSD, currently at over 800 members
  \item Hosted technical workshops on a wide variety of topics, including full
    stack web development (\textbf{React}, \textbf{Node.js}, \textbf{MongoDB}), \textbf{ROS}, and \textbf{Haskell}
\end{entry}


%\header{Awards}
%2021-2022 ECE Student Service Award

%==== Hobbies ====%
\begin{comment}
\header{Favorite Video Games}
Street Fighter 6, Tears of the Kingdom, Hollow Knight, Celeste
\end{comment}
\ 
\end{document}

%%% Local Variables:
%%% mode: LaTeX
%%% TeX-master: t
%%% End:
