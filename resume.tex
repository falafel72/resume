\documentclass[a4paper]{article}
    \usepackage{fullpage}
    \usepackage{amsmath}
    \usepackage{amssymb}
    \usepackage{textcomp}
    \usepackage[utf8]{inputenc}
    \usepackage[T1]{fontenc}
    \textheight=9in
    \pagestyle{empty}
    \raggedright
    \usepackage[left=0.4in,right=0.4in,bottom=0.3in,top=0.3in]{geometry}
    \usepackage{comment}
    \usepackage{hyperref}
    \usepackage{fontawesome}
    \usepackage{etoolbox}
    \usepackage{multibib}
    \newcites{pat}{Patents}
    \patchcmd{\thebibliography}{\section*{\refname}}{}{}{}


   \hypersetup{
      colorlinks=true,
      linkcolor=blue,
      filecolor=magenta,
      urlcolor=black,
    }

    \makeatletter
    \renewcommand\@biblabel[1]{}
    \makeatother
    %\renewcommand{\encodingdefault}{cg}
%\renewcommand{\rmdefault}{lgrcmr}

\def\bull{\vrule height 0.8ex width .7ex depth -.1ex }

% DEFINITIONS FOR RESUME %%%%%%%%%%%%%%%%%%%%%%%

\newcommand{\area} [2] {
    \vspace*{-9pt}
    \begin{verse}
        \textbf{#1}   #2
    \end{verse}
}

\newcommand{\lineunder} {
    \vspace*{-8pt} \\
    \hspace*{-18pt} \hrulefill \\
}

\newcommand{\header} [1] {
    {\hspace*{-18pt}\vspace*{6pt} \textsc{#1}}
    \vspace*{-6pt} \lineunder
}

\newcommand{\employer} [3] {
    { \textbf{#1} (#2)\\ \underline{\textbf{\emph{#3}}}\\  }
}

\newcommand{\contact} [3] {
    \vspace*{-10pt}
    \begin{center}
        {\Huge \scshape {#1}}\\
        #2 \\ #3
    \end{center}
    \vspace*{-8pt}
}

\newenvironment{achievements}{
    \begin{list}
        {$\bullet$}{\topsep 0pt \itemsep -2pt}}{\vspace*{4pt}
    \end{list}
}

\newcommand{\schoolwithcourses} [4] {
    \textbf{#1} #2 $\bullet$ #3\\
    #4 \\
    \vspace*{5pt}
}

\newcommand{\school} [4] {
    \textbf{#1} #2 $\bullet$ #3\\
    #4 \\
}

\newenvironment{entry}[4][]{
  \textbf{#2} \hfill #1 \\
  \textit{#3} \hfill #4 \\
  \vspace{-2mm}
  \begin{itemize} \itemsep 0em
  }
  {
  \end{itemize}
}
% END RESUME DEFINITIONS %%%%%%%%%%%%%%%%%%%%%%%

\begin{document}
%\vspace*{-40pt}

    

%==== Profile ====%
%\vspace*{-10pt}
\begin{center}
	{\Huge \scshape {Kyle Hu}}\\
  \vspace{2mm}
	\faGlobe\:San Diego, CA
  $\cdot$ \faEnvelope\:kylehu2000@gmail.com
  $\cdot$ \faLinkedinSquare\:\href{https://www.linkedin.com/in/kshu082/}{kshu082}
  $\cdot$ \faGithubSquare\:\href{https://www.github.com/falafel72}{falafel72} \\
\end{center}

\begin{comment}
\header{Objective}
Looking for an entry-level full time job where I can fully utilize my
robotics education and systems experience.
%\vspace{-1mm}
\end{comment}
% ==== Education ====%
\header{Education}
\vspace{1mm}
\textbf{University of California, San Diego}\hfill San Diego, CA\\
MS Intelligent Systems, Robotics and Control \textit{GPA: 3.63} \hfill Sep 2022 - Jun 2024\\
\vspace{1mm}
\textit{Coursework} \\
Statistical Learning | Sensing and Estimation in Robotics | Planning and Learning in Robotics | Advanced Linear Algebra | Dynamical Systems | Random Processes | Computer Vision | GPU Programming | Ocean Optics\\
\vspace{2mm}
\textbf{University of California, San Diego}\hfill San Diego, CA\\
BS Computer Engineering \textit{GPA: 3.77} \hfill Sep 2018 - Jun 2022\\


\vspace{2mm}

% ==== Skills ====%
\header{Skills}
\vspace{2mm}
\begin{tabular}{ l l }
	Programming Languages: & C/C++, Python, Java, MATLAB, JavaScript, Haskell, Rust  \\
  Software Tools:       & Git, Bash, Linux, NumPy, Matplotlib, Jupyter Notebook, ROS, OpenCV, HTML, CSS, SCSS, \\
   & CGAL, Qt, Swing, Docker, GitHub Actions, clap, serde, TOML, JSON, VSCode, Vim, Emacs \\
  Embedded Skills: & STM32, I2C, UART, SPI, DMA, HAL\\
  Hardware Tools: & Multimeter, Oscilloscope, Soldering, Solidworks, 3D Printing, Laser Cutting \\
	Spoken Languages:      & English, Cantonese, Mandarin                     \\
\end{tabular}
\vspace{2mm}

%==== Research ====%
\header{Research}
\vspace{1mm}
\begin{entry}[San Diego, CA]{Engineers for Exploration}{Research
    Assistant, Project Co-lead}{February 2022 - Present}
	\item Designing the first citizen science platform for gathering fish population data
    using off-the-shelf components
    \item Pivoted from a previously unsuccessful stereo-camera approach to a simpler single camera and laser system
    \item Managing team in excess of 10 students to aid in engineering and research efforts
  \item Derived mathematical calibration procedure using \textbf{Gauss-Newton} optimization
    for multiple sensor alignment
  \item Planned and conducted experiments to iterate on in-field calibration procedures and verify system accuracy
	\item Architected and developed software pipeline in \textbf{Python} to automate image
    processing and sensor calibration
\end{entry}

%==== Publications ====%
\header{Publications}
\nocite{thesis,paper}
\bibliographystyle{IEEEtran}
\bibliography{publications}

%==== Patents ====%
\header{Patents}
\nocitepat{fishsense-patent}
\bibliographystylepat{IEEEtran}
\bibliographypat{patents}

%==== Experience ====%
\header{Work Experience}
\vspace{1mm}
\begin{entry}[San Diego, CA]{MaXentric Techologies}{Engineering Intern}{June
    2022 - September 2022}
	\item Implemented high-level digital signal processing algorithms in \textbf{C} for a radar-based non-contact vitals monitor
	\item Heavily refactored existing codebase to be modular and
    extensible, unit testing against \textbf{MATLAB} implementation
  \item Increased speed of convergence for estimate of heart and respiration
    rate to ground truth by 50\%
\end{entry}

  \begin{entry}[San Diego, CA]{Amazon}{Software Development Engineering
      Intern}{June 2021 - September 2021}
  \item Designed and implemented a favorites list widget on the grocery cart page using \textbf{AWS} and \textbf{Spring}
	\item Performed cost/benefit analysis to identify appropriate \textbf{AWS} services (Lambda w/ Dynamo DB Table)
	\item Implemented new endpoints on microservices used in production code to host front-end
\end{entry}

\begin{entry}[San Diego, CA]{HP}{Software Engineering Intern}{June 2020 -
    September 2020}
\item Built an internal 3D measurement tool for biometrics in \textbf{C++} using \textbf{CGAL}, \textbf{VTK} and \textbf{Qt}
\item Implemented multi-threaded geodesic path searching, and employed
  Levenberg-Marquardt for model alignment
\item Digitized the measurement pipeline for custom 3D insoles, reducing time from several hours to several minutes
\end{entry}
\pagebreak

%==== Communication ====%
\header{Leadership and Communication}
\vspace{1mm}

\begin{entry}{ECE 15 (Engineering Computation)}{Teaching Assistant}{September 2022 - March 2024}
  \item Guided students through fundamentals of \textbf{C},
    including variables, control flow, pointers, and process memory
  \item Held office hours, provided assistance with assignments, and graded exams for up to 250 students per class
\end{entry}

\begin{entry}{CSE 100 (Advanced Data Structures)}{Tutor}{September 2020 - March 2021}
  \item Taught data structures including red/black trees, multiway tries, and disjoint sets
  \item Held office hours and provided assistance with assignments
\end{entry}

\begin{entry}{ACM UCSD}{Cofounder \& Technical Director}{May 2019 - June 2021}
  \item Co-founded the largest engineering organization at UCSD, currently at over 800 members
  \item Hosted technical workshops on a wide variety of topics, including full
    stack web development (\textbf{React}, \textbf{Express}, \textbf{Node.js}, \textbf{MongoDB}), \textbf{ROS} (Melodic), and \textbf{Haskell}
\end{entry}

% ==== Projects ====%
\header{Projects}
\vspace{1mm}
\begin{entry}{E4E Data Management Tool}{Software}{October 2023 - Present}
\item Porting a data upload and validation tool written in \textbf{Python} to \textbf{Rust} for simpler distribution and better performance
\item Writing command line interface using \textbf{clap}, writing configuration state to disk using \textbf{serde}
\item Conducting code reviews with team of 3, tracking changes using Git feature branch workflow
\end{entry}

\begin{entry}{Visual SLAM}{Software}{March 2022}
\item Implemented simultaneous localization and mapping using stereo video and \textbf{IMU} data
\item Implemented an \textbf{extended Kalman filter} to accurately predict the car position based on camera observation model.
\item Written in \textbf{Python} and \textbf{Jupyter Notebook} using pure \textbf{numpy}, plotted results with \textbf{matplotlib}
\end{entry}

\begin{entry}{Pinball Machine}{Rapid Prototyping, Computer Aided Design, Circuit
    Design}{April 2022 - June 2022}
\item Modeled all components in \textbf{Solidworks}, manufactured primarily with \textbf{3D printing} and \textbf{laser cutting}
\item Designed circuitry to control flippers on button press, and bumpers that
  trigger (via piezoelectric) when hit
\item Designed state machine to run on an \textbf{Arduino} to control all actuators and represent game state
\end{entry}

\begin{entry}{Serial Driver}{Firmware}{February 2022 - June 2022}
\item Part of sleep timer firmware to drive a low power radio telemetry system
  for tracking collared lizards
\item Implemented a serial driver for an \textbf{STM32} chip to facilitate
  communication over \textbf{UART}
\item Implemented ring buffer abstraction for robust processing of data from multiple serial ports concurrently
\item Ring buffer memory updated using \textbf{DMA} triggered by \textbf{UART}
  interrupt to store data as quickly as possible
\end{entry}

\begin{entry}{Yonder Deep - Autonomous AUV}{Software}{October 2021 - June 2022}
\item Developed software for an autonomous underwater vehicle, designed to study glacier melting using hydrophones
\item Wrote subroutines for diving and heading correction in \textbf{Python}, using \textbf{IMU}, pressure sensor, and \textbf{PID} control loops
\end{entry}

\begin{entry}{Checkup}{Back End Development}{March 2021}
\item Winner of Best Overall Hack at SF Hacks 2021
\item Telemedicine website that takes basic diagnostic measurements using only the user’s camera and microphone
\item Implemented heart rate measurement through webcam with a \textbf{Fourier Transform}, using \textbf{JavaScript}
\end{entry}

\begin{entry}{Campus Friends}{Back End Development}{August 2020 - March 2021}
 \item Mobile app designed to help users find friends online
\item Implemented signaling server in \textbf{Starlette} with private video chat room functionality through \textbf{WebRTC}
\item Stored call data (start and end times, users involved) with \textbf{SQL} database (SQLAlchemy)
\end{entry}

\begin{entry}{Carbon Price Tag}{Full Stack Web Development}{February 2020}
\item \textbf{Chrome extension} that displays extra carbon cost of items from Amazon Whole Foods
\item Implemented shipping route calculation with \textbf{Google Maps API} and used
  \textbf{Flask} backend
\item Winner of the Sustainability vertical at HackSC 2020
\end{entry}

\begin{entry}{UCSD IEEE Project Drive}{Software, Firmware, Circuit Design}{September 2018 - March 2020}
\item Developed and trained a convolutional neural network for an \textbf{Intel Realsense D455}-based autonomous vehicle
\item Built electronics to manage power delivery from LIPO battery to ESC, \textbf{Arduino} controller and \textbf{Jetson TX2}
\item Maintained \textbf{Ubuntu} \textbf{Docker} image to streamline collaborative development by standardizing development system
\item Created \textbf{ROS} (Melodic) nodes to facilitate \textbf{LiDAR} and \textbf{Realsense} data ingest and processing
\end{entry}

\begin{entry}{ACM Meme Gen}{Full Stack Web Development}{June 2019 - December 2019}
\item Created a meme generator as a supplement for students to learn \textbf{MongoDB}, \textbf{Express}, \textbf{React}, and \textbf{Node.js}
\item Implemented \textbf{React} components and state management, called the
  Imgflip API for templates
\item \textbf{MongoDB} back-end used to store user-created memes for a central gallery
\end{entry}

\begin{entry}{Infantry Firmware - Robomaster}{Firmware, Circuit Design}{January 2019 - August 2019}
\item Implemented control firmware on \textbf{STM32} chip for a 2-DOF turret mounted on a 4-wheel robot
\item Angle control of motors achieved using sensor data from \textbf{encoders} and \textbf{IMU}
  over \textbf{UART} and \textbf{I2C} with \textbf{PID}
\item Designed and soldered power circuit with supercaps in parallel with
  battery to circumvent power restrictions
\end{entry}

\begin{entry}{Reef Pin}{Firmware, Circuit Design}{November 2018 - June 2019}
\item Designed and tested system for a device used to measure depth for coral reef mapping
\item Used a MS5803-14BA pressure sensor to measure depth and an MLX90393 magnetometer for cardinal direction
\item Used \textbf{I2C} to retrieve sensor data and communicate with LED driver, logged data to an SD card using \textbf{SPI}
\end{entry}

%==== Publications ====%

\header{Awards}
\begin{achievements}
\item 2024 Jacobs Research Expo Transdisciplinary Collaboration Award
\item 2021-2022 ECE Student Service Award
\end{achievements}

%==== Hobbies ====%
\begin{comment}
\header{Favorite Video Games}
Street Fighter 6, Tears of the Kingdom, Hollow Knight, Celeste
\end{comment}
\ 
\end{document}

%%% Local Variables:
%%% mode: LaTeX
%%% TeX-master: t
%%% End:
